\documentclass[11pt, a4paper]{article}
\usepackage{amsmath, amsthm, amssymb, calrsfs, wasysym, verbatim, bbm, color, graphics, geometry}
\usepackage[utf8]{inputenc} % comment when using lualatex
\usepackage[italian]{babel} % lingua e a-capo-sillabato
\usepackage{fullpage}
\usepackage{graphicx}
%\usepackage[hidelinks]{hyperref,xcolor} % link di pagina
\usepackage[bottom]{footmisc} % note appiccicate al fondo della pagina
\usepackage{float} % per posizionamento immagini

\geometry{tmargin=.75in, bmargin=.75in, lmargin=.75in, rmargin = .75in}  

\newcommand{\R}{\mathbb{R}}
\newcommand{\C}{\mathbb{C}}
\newcommand{\Z}{\mathbb{Z}}
\newcommand{\N}{\mathbb{N}}
\newcommand{\Q}{\mathbb{Q}}
\newcommand{\Cdot}{\boldsymbol{\cdot}}

\newtheorem{thm}{Theorem}
\newtheorem{defn}{Definition}
\newtheorem{conv}{Convention}
\newtheorem{rem}{Remark}
\newtheorem{lem}{Lemma}
\newtheorem{cor}{Corollary}



\definecolor{dkgreen}{rgb}{0,0.6,0}
\definecolor{gray}{rgb}{0.5,0.5,0.5}
\definecolor{mauve}{rgb}{0.58,0,0.82}


\title{Network Infrastructures }
\author{Raffaele Castagna}

\date{Academic Year 2025-2026}

\begin{document}

\maketitle
\tableofcontents
\newpage
\section{Network functional areas}
An \textbf{Access network} is part of a communications network which connect subscribers to their service provider, to actually connect them historically ISPs have used copper lines which also carried phone signals, but they have now heavily invested into optical fiber cables.\\\\
The \textbf{Core network} is a backbone network, it has always used fiber optic cables and it is constitued by the optical backbone, we could say that the internet is a giant core network, since it consists of many service providers that run their own core networks, which are interconnected.\\\\
The \textbf{Edge network} is located between acess and core, it used to decide which path the packets should take using MPLS (multiprotocol Label Switching), It is used to decide the service the user wants to use, but it's also used to run
services that in the past where run in the cloud, they need to run closer to the
user, for example. The closer they run, the least the delay.
So the edge is fundamental to distinguish the kind of service we use, but also
because it hosts computing servers (known as edge computing)\\\\
We have different types of accesses:
\begin{itemize}
    \item \textbf{Wired access}, high reliability, high speed, low latency
    \item \textbf{Wireless access}, mobile, flexibile and easy to setup.
    \item \textbf{Satellite Access} Wide coverage, suitable for remote areas.
    \item \textbf{Fiber Optic Access} uses fiber optics (so costly) but high speed and low latency.
    \item \textbf{DSL Access} Widespread and cost effective
    \item \textbf{Cable Access} uses coaxial cables, high speed and widely available, suitable for TV
\end{itemize}
and many more...\\\\
\subsection{FTTX}
With fiber becoming widespread we have what we call FTTX (fiber to the X) where X can be:
\begin{itemize}
    \item \textbf{FTTH} Fiber to the home
    \item \textbf{FTTB} Fiber to the building
    \item \textbf{FTTC} Fiber to the curb
    \item \textbf{FTTN} Fiber to the node
    \item \textbf{FTTP} Fiber to the premises
    \item \textbf{FTTD} Fiber to the desk
    \item \textbf{FTTU} Fiber to the user
    \item \textbf{FTTO} Fiber to the office
    \item \textbf{FTTZ} Fiber to the zone
\end{itemize}
In fttx we have multiple elements, the OLT (optical line terminal) the ONU (Optical Network Unit) , the ONT (Optical Network Termination) and the ODN, Optical Distribution Network.
\begin{center}
    \includegraphics[scale=0.5]{img/AccessNetworks/FTTX/reference.png}
\end{center}
These can be used in multiple ways to create an infrastructure, one example is AON (Active Optical Network), where the loops represent the fiber. This topology is not used as the cost from local exchange to user is pretty high, as fiber has a high labor cost. 70€/m \\
A possible solution is a single fiber to a single element, called active element which shortens the path you need to dig. The active node receives light and then converts it into electricity and then converts it into another light signal.
\begin{center}
    \includegraphics[scale=0.5]{img/AccessNetworks/FTTX/AON.png}
\end{center}
In the fiber domain we have a more favorable solution, the PON (Passive optical network),where instead of an active node we have an optical power splitter/combiner which does not use electricity, in this case when we split, the same signal arrives to everyone, but only the ONU/ONT that recognizes its address will process the signal, the others will discard it. This is a more cost effective solution as we don't need electricity, but the signal weakens as we split it, so we need to be careful about how many splits we do.
\begin{center}
    \includegraphics[scale=0.5]{img/AccessNetworks/FTTX/PON.png}
\end{center}
We also have fiber to the exchange, where fiber terminates to the central office and is then connected via copper (e.g. ADSL) (VDSL is very high speed digital subscriber line)
\begin{center}
    \includegraphics[scale=0.5]{img/AccessNetworks/FTTX/FTTX.png}
\end{center}
We also have FTTC which is fiber to the curb, or more commonly fiber to the cabinet, where the fiber terminates near the building, and then the last mile is done via copper (e.g. VDSL).\\
After that we have the various FTTP/FTTB/FTTH.
\begin{center}
    \includegraphics[scale=0.5]{img/AccessNetworks/FTTX/FTTP.png}
\end{center}
\subsection{Wireless Access Networks}
Sometimes it isn't economical to dig/ it isn't worth the trouble, so we use wireless access networks, which are very complicated.
Areas are divided into cells, in each cell we find an antenna (base station), antennas are also connected to the backhole (in the past made with copper).\\
\textit{Note:}Availability means localizing the user and then managing the information regarding the movement of the user
\begin{center}
    \includegraphics[scale=0.5]{img/AccessNetworks/Wireless/2G.png}
\end{center}
We also have 5g networks, where we have the User equipment, the antenna is called the gNodeB, the access and mobility management function AMF, session Management Function SMF, and User Plane Function UPF, the User Plane is both hardware and software and the Control Plane is software, the antenna is hardware ofc.
\begin{center}
    \includegraphics[scale=0.5]{img/AccessNetworks/Wireless/5G.png}
\end{center}
\subsection{Satellite Access Networks}
Around ~500-2000 km above earth with a short orbital period, it gives low latency communications, therefore it requires a lot of satellites, they utilize the Ku and Ka Band.\\
So the advantages are low latency, high bandwidth potential and it gives us global coverage, but the challenges we face are the large amount of satellites needed, there's a high deployment and maintenance costs, along with space debris and a short satellite lifespan.\\
\subsection{XDSL}
Dsl stands for digital subscriber line (subscriber of a contract), it has many different standards:
\begin{itemize}
    \item ISDL 
    \item HDSL
    \item SDSL
    \item ADSL
    \item ADSL lite
    \item RADSL
    \item VDSLV
    \item VDSL2
\end{itemize}
In the first iterations the telephone distribution network featured a twisted pair for each user, connecting them to the
CO. Bundles of twisted pairs from various users converged at a cabinet and were aggregated to
the CO. Early DSL iterations like ISDN (the first DSL) doubled the connections by aggregating
two twisted pairs to enable one channel for voice and another for data. However, However, the
fact that the bandwidth used by the user for data transmission is the same as that used by voice
traffic (4 kHz) soon became a bottleneck.
\begin{center}
    \includegraphics[scale=0.5]{img/AccessNetworks/XDSL/Schematic.png}
\end{center}
Analog modems represented a foundational method for delivering data services via the local
telephone network. These devices were capable of transmitting data at speeds of 56 Kbps,
effectively converting digital data into analog signals suitable for transmission over standard
telephone lines. The same connection could alternate between voice and data transmission,
though not simultaneously.
To achieve data transfer, voice modems utilized the voice frequency band, essentially
simulating a phone call to relay information. Once transmitted, the central office (C.O.) detected
and processed this signal as data. Each data session incurred charges equivalent to a standard
phone call, making prolonged usage financially burdensome.
\begin{center}
    \includegraphics[scale=0.5]{img/AccessNetworks/XDSL/Shannon.png}
\end{center}
Even after we stopped using the 4 khZ band we still had limitations that had to do with the space inside the C.O. and the fact that the performance of copper degraded with long cables.\\
\subsection{ADSL}
ADSL (Asymmetric Digital Subscriber Line) is a type of DSL technology that provides high-speed internet access over traditional copper telephone lines. It is called "asymmetric" because it offers different speeds for downloading and uploading data, with download speeds typically being much higher than upload speeds. This asymmetry is designed to accommodate the typical usage patterns of internet users, who often download more data than they upload.\\\\
ADSL utilizes higher frequency bands on the copper line, but this introduces more interference (a signal overlaps in time and frequency) and attenuation, to counteract this ADSL uses multiple such as frequency division and eco cancellation, in frequency division we divide the frequency band into two parts, one for upstream data and one for downstream data. In echo cancellation we use the same frequency band for both upstream and downstream data, but we use a technique called echo cancellation to separate the two signals.\\
\begin{center}
    \includegraphics[scale=0.5]{img/AccessNetworks/XDSL/Frequency.png}
\end{center}
We can formalize this via FEXT (far end cross-talk), the cross talk between transmitter and a receiver placed on opposite sides of the cable. You can measure interference at the receiver side.
\begin{center}
    \includegraphics[scale=0.5]{img/AccessNetworks/XDSL/FEXT.png}
\end{center}
We also have NEXT where the receiver is near the transmitter and we have the same problem.
\begin{center}
    \includegraphics[scale=0.5]{img/AccessNetworks/XDSL/NEXT.png}
\end{center}
If we were using the same type of connection (down) we may have interference, but remember that interference in calculated at the receiver side, plus we have interference only if the cables are near (centimeters), this may also happen if we use echo cancellation and occupy part of the upstream.
By transmitting downstream data within the upstream band, the C.O. can apply echo
cancellation.\\
The C.O. knows its downstream transmissions and the upstream data it receives, enabling
it to remove interference caused by overlapping flows.\\
This technique effectively increases the downstream data rate by omitting the upstream
portion when necessary.\\\\
When we have both a telephone and ADSL line, we may have problems when dialing a phone call, to forego this problem we installed filters to separate the communications.\\\\
We also may have signal attenuation, which if we know it and it is constant, we can amplify the signal. The rule is that if you have a short bandwidth then the attenuation is constant, if it isn't short, it isn't costant, to solve this problem you split the bandwidth into different channels with low bandwidth, but since we can transmit simultaneously then we have around the same speed.\\
This technique is called modulation and there exist multiple types:
\begin{itemize} 
    \item \textbf{CAP} Carrierless Amplitude/Phase modulation, it uses two orthogonal carriers (sine and cosine) to transmit data, it is simple and cost effective, but it is sensitive to noise and interference. Also we havent really done it.
    \item \textbf{DMT} Discrete Multi-Tone modulation, it divides the available bandwidth into multiple sub-channels (tones), each tone can be modulated independently, it is more complex and expensive, but it is more robust against noise and interference.
\end{itemize}
The one we focus on is DMT, a graph of how DMT works is shown below:
\begin{center}
    \includegraphics[scale=0.5]{img/AccessNetworks/XDSL/DMT.png}
\end{center}
Each sub-band has a distinct carrier frequency,allowing simultaneous data transmission. Key features include:
\begin{itemize}
    \item \textbf{Band Division}: The band is segmented into 249 sub-channels for downstream and 25 for upstream, each 4 KHz wide, resulting in an overall bandwidth of 1 MHz.
    \item \textbf{Frequency-Specific Modulation}: Modulation is adapted to the attenuation characteristics of each sub-channel. This prevents data transmission over heavily attenuated (or "unlucky") frequencies, improving overall efficiency.
    \item \textbf{Channel Behavior}:  Sub-channel behavior is nearly constant, enabling higher SNR (Signal-to-Noise Ratio) within each sub-channel. This flat behavior ensures optimal modulation using Shannon's law.
    \item \textbf{Dynamic Adaptation}: DMT dynamically adjusts the data rate to line conditions, offering resilience against interference by reallocating resources from imapired sub-channels to better-performing ones.
    \item \textbf{Adaptive Modulation}:  Each sub-channel employs a QAM modulation scheme suited to its specific SNR conditions. Higher SNR sub-channels use denser QAM constellations (e.g., QAM 64, QAM 256), while lower SNR channels use simpler schemes (e.g., QAM 16 or BPSK).
\end{itemize}
The Water filling Algorithm is a technique used in ADSL modems to adapt QAM modulation to the specific conditions of the telephone line. It optimizes data trasmission by considering:
\begin{itemize}
    \item \textbf{SNR Dependency}: Higher SNR sub-channels (deeper water) can support denser QAM modulations, enabling higher data rates.
    \item \textbf{Dynamic Adjustment}: Each sub-channel is individually analyzed, and the number of bits per symbol is adjusted according to its SNR.
    \item \textbf{Noise management}:Channels with higher noise (shallow water) receive lower power and simpler modulation schemes to minimize errors.
    \item \textbf{Power Allocation}: Available power is distributed to maximize the overall bit rate while considering the unique conditions of each sub-channel.
\end{itemize}

Let's talk about the architecture of ADSL:
\begin{center}
    \includegraphics[scale=0.5]{img/AccessNetworks/XDSL/architecture.png}
\end{center}
Its key components include:\\
\textbf{ATU-R} (User Modem): Located at the user's premises\\
\textbf{ATU-C} (C.O. Modem): Located at the Central Office (C.O.), with one modem for each user. The C.O. houses hundreds of modems, requiring substantial space and energy, a significant disadvantage of this architecture. For this reason, the DSLAM (DSL Access Multiplexer) was implemented: a device that worked as multiple ATU-C together that had the main purpose of occupying less space in the C.O
\textbf{Splitter/Filter}: Separates voice (POTS) and data signals, allowing simultaneous use of both services, it's present at both the CO and the end user's side.\\
PPP is designed to provide a reliable communication link between two peers over a simple,
direct connection. It is specifically used in ADSL to connect the end user to the Central Office (CO), and it offers important functions such as:
\begin{itemize}
    \item Authentication
    \item Authorization
    \item Automatic configuration of network interfaces
\end{itemize}
\begin{center}
    \includegraphics[scale=0.5]{img/AccessNetworks/XDSL/PPP.png}
\end{center}
PPP is made up of two main sub-protocols:
\begin{itemize}
    \item \textbf{LCP} (Link Control Protocol): Responsible for establishing, configuring, and terminating the data link connection. It negotiates link parameters and ensures the link is operational.
    \item \textbf{NCP} (Network Control Protocol): NCP is a family of protocols that configure network parameters for the network layer, adapting to different network protocols
\end{itemize}
\subsection{VDSL}
\begin{center}
    \includegraphics[scale=0.5]{img/AccessNetworks/VDSL/VDSL.png}
\end{center}
To actually improve ADSL we have VDSL (Very high bit-rate DSL) where to have more broadband we use higher frequencies, while also trying to get as close to the user with fiber as possible (not anymore at the CO, but a the user Node), so the DSLAMs,ATU-C and splitters are in the cabinets\\
We have 2 main types of configurations for VDSL:
\begin{itemize}
    \item \textbf{Frequency Division Duplexing (FDD)}: We ahve a band for upstream and downstream, while also providing another for upstream to have a symmetric connection.
    \item \textbf{Time Division Duplexing (TDD)}: We use the same band for upstream and downstream, but we divide it in time slots, so we can have a symmetric connection.
\end{itemize}
\begin{center}
    \includegraphics[scale=0.5]{img/AccessNetworks/VDSL/types.png}
\end{center}
To cancel crosstalk we use vectoring , a technique which uses an anti-sginal methodology to enhance signal quality and network performance and its done by doing the follwing stepsL
\begin{itemize}
    \item \textbf{Crosstalk Detection}: The system continuously monitors the signals on each twisted pair to identify crosstalk interference.
    \item \textbf{Anti Signal Calculation}: We create a signal with opposite phase to the crosstalk
    \item \textbf{Inejction of Anti-Signal}: The anti-signal is injected into the line to cancel out the crosstalk, while ensuring synchronization for maximum effectiveness.
    \item \textbf{Complete Synchronization}: Precise synchronization of all lines is essential for effective vectoring
    \item \textbf{Continuous Monitoring and Adjustment}: The system continuously monitors the lines and adjusts the anti-signals as needed to adapt to changing conditions.
\end{itemize}
\begin{center}
    \includegraphics[scale=0.5]{img/AccessNetworks/VDSL/vectoring.png}
\end{center}
Vectoring does give us increased speeds, stability and its actually pretty cheap, but it requires coordination, computational power and the infrastructure itself isn't cheap.
\subsection{PON infrastructure}
In optical fiber communication we utilize a laser to trasmit information, like copper cables we still have some attenuation, but with the newer generation of fiber optic cables, we can utilize multiple wavelengths to trasmit data.\\
At the core of our PON we have passive optical splitters, which split the light into multiple directions, in each splitter we have an end that is closed off and by daisy chaining multiple splitters we can achieve a 1XN splitter, but every time we split the signal we reduce the signal strength by $10log_{2}(N)$ dB, where N is the number of outputs.\\
\begin{center}
    \includegraphics[scale=0.5]{img/AccessNetworks/PON/operations.png}
\end{center}
We transmit a light (wavelength), the configuration is done by having our usual downstream and upstream, we have our Optical Line Terminal (used to transmit information) and in the building we have our ONT, each wavelength is physically separated (1 for up and 1 for down), we can have 10s of kms made of fiber optic since the signal doesn't degrade as much as copper.\\\\
The passive splitter broadcasts the signal to everyone, but we do append an address so that only the intended receiver actually utilizes the signal, we still do actually need to cypher the data so that only the receiver can actually act on the data. Though even with that we can actually see how much data someone is receiving.\\\\
\begin{center}
    \includegraphics[scale=0.5]{img/AccessNetworks/PON/downstream.png}
\end{center}
We still have the problem that upstream connections share the same wavelength so we may have collisions, since ONUs cannot listen to other ONU and splitters are passive we need a way to actually transmit data upstream. We can solve this by providing a schedule for each user to transmit data, this is done with Time Division Multiplexing Access (\textbf{TDMA}),but still that does impact our bandwidth.\\\\
\begin{center}
    \includegraphics[scale=0.5]{img/AccessNetworks/PON/upstream.png}
\end{center}
Remember though that since everyone is at different distances fromm the splitter our signal does get attenuated, therefore the OLT receives signals at different powers and must utilize automatic gain control to get a “clear” signal.
\begin{center}
    \includegraphics[scale=0.5]{img/AccessNetworks/PON/AGC.png}
\end{center}
Ethernet PON (EPON) introduces the use of Ethernet frames, specifically 802.3ah standard
frames, in PON systems. This ensures compatibility with existing Ethernet networks. Each
packet in the EPON network contains a header field with the MAC address of the intended
recipient, ensuring that only the relevant packets are received by the correct ONU.
\begin{center}
    \includegraphics[scale=0.5]{img/AccessNetworks/PON/EPON_down.png}
\end{center}
As with traditional PON systems, the ONUs transmit their upstream data in synchronized time
slots. The OLT ensures that the signals from different ONUs reach it with equal power,
maintaining the same level of intensity for each transmission.
\begin{center}
    \includegraphics[scale=0.5]{img/AccessNetworks/PON/EPON_up.png}
\end{center}
\end{document}